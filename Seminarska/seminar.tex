
\documentclass[ letterpaper, titlepage, fleqn]{article}

\usepackage[utf8]{inputenc}
\usepackage[slovene]{babel}
\usepackage[margin=60px]{geometry}
\usepackage{amsmath}
\usepackage{amssymb}
\usepackage{enumerate}
\usepackage{graphicx}
\usepackage{mathrsfs}
\usepackage{mathabx}
\setlength\parindent{0pt}

\newcommand{\R}{\mathbb R}
\newcommand{\N}{\mathbb N}
\newcommand{\Z}{\mathbb Z}
\newcommand{\C}{\mathbb C}
\newcommand{\Q}{\mathbb Q}
\newcommand{\F}{\mathscr{F}}
\newcommand{\E}{\mathbb E}
\newcommand{\FF}{\mathbb F}
\newcommand{\K}{\mathbb K}
\newcommand{\D}{\mathbb D}

\newcommand{\mul}{\text{mul}}
\newcommand{\add}{\text{add}}
\newcommand{\abs}{\text{abs}}
\newcommand{\aph}{\text{@}}
\newcommand{\primea}{\textsc{\char13}}
\newcommand{\norm}[1]{\left\lVert#1\right\rVert}
\newcommand{\scalar}[1]{\left\langle#1\right\rangle}
\newcommand{\openbox}{\leavevmode
  \hbox to.77778em{%
  \hfil\vrule
  \vbox to.675em{\hrule width.6em\vfil\hrule}%
  \vrule\hfil}}

\begin{document}

\thispagestyle{empty}
\noindent{\large
UNIVERZA V LJUBLJANI\\[1mm]
FAKULTETA ZA MATEMATIKO IN FIZIKO\\[5mm]
\vfill

\begin{center}{\large
Nejc Ševerkar, Matija Šteblaj\\[2mm]
{\bf Spojena kubična Bezierjeva krpa}\\[10mm]
RPGO}
\end{center}
\vfill

\noindent{\large
Ljubljana, 2021}
\pagebreak

\thispagestyle{empty}
\tableofcontents
\pagebreak

\section{Interpolacija razsevnih podatkov v prostoru}
Prejšnjo metodo želimo uporabiti na problemu interpolacije točk $P = (p_i)_{i=1}^n$, kjer je $p_i = (x_i,y_i,z_i) \in \R^3$. 
Torej mislimo si, da te podatki ležijo na grafu neke zvezno odvedljive funkcije $f$, ki pa je seveda ne poznamo.
Spomnimo se, da metoda Goodman-Said zahteva poleg vrednosti še parcialne odvode prvega reda v točkah $(x_i,y_i)_{i=1}^n$.
Ker teh nimamo, jih moramo oceniti.

\subsection{Aproksimacija parcialnih odvodov}
Recimo, da ocenjujemo odvoda v testni točki $p_k \in P$ (natančneje je to odvod $f$ v $(x_k,y_k)$). 
To bomo storili v treh korakih.

\begin{enumerate}
\item Za oceno odvoda v $p_k$  bomo seveda potrebovali neke informacije o vrednostih $f$ 
v točkah blizu $(x_k,y_k)$. Vse kar imamo na voljo so točke $(x_i,y_i)$, torej izmed njih
izberemo tiste, ki so $(x_k,y_k)$ dovolj blizu. To naredimo z izborom radija $r_k$ in obravnavo točk $p_j \in P$, 
za katere velja
$$d((x_j,y_j), (x_k,y_k)) = d^k_j \in (0, r_k].$$
Označimo množico indeksov teh z $J_k$.
\item Ker tudi med izbranimi točkami prioritiziramo tiste, ki so naši testni točki bližje,
jih ustrezno utežimo. Za $j \in J_k$ definiramo
$$w^k_j := \frac{r_k - d^k_j}{r_k \cdot d^k_j},$$
utež točke $p_j$ glede na $p_k$. 
\item Za $p_k$ definirajmo interpolacijski polinom druge stopnje kot
$$p(x,y) := z_k + a (x - x_k)^2 + b (x - x_k) \cdot (y - y_k) + c (y - y_k)^2 + d (x-x_k) + e(y-y_k),$$
kjer so $a,b,c,d,e \in \R$ nedoločeni koeficienti in velja 
$$p_x(x_k,y_k) = d \quad \text{in} \quad p_y(x_k,y_k) = e.$$
Ti vrednosti bosta oceni za parcialna odvoda v točki $p_k$.
Da bo to smiselno, mora ta polinom v okolici $(x_k,y_k)$ dobro aproksimirati funckijo $f$, torej $(x_j,y_j)$ za $j \in J_k$.
Če upoštevamo še uteži posamezne točke, so vrednosti doloćene z minimizacijskim problemom
$$\sum_{j\in J_k} (w^k_j \cdot (p(x_j,y_j) - z_j))^2 = \norm{W_k \cdot Au - W_k \cdot v}^2,$$
kjer so za $J_k = \{j_1, j_2, \dots, j_{n_k}\}$ \\
\begin{equation*}
\begin{aligned}
W_k = &
\begin{bmatrix}
w_{j_1}^k &  &  &  \\
& w_{j_2}^k &  &  \\
& & \ddots & \\
&  &  & w_{j_{n_k}}^k
\end{bmatrix},
\quad
v = 
\begin{bmatrix}
z_{j_1}- z_k \\
z_{j_2} - z_k \\
\vdots  \\
z_{j_{n_k}} - z_k
\end{bmatrix}, \quad
u =
\begin{bmatrix}
a \\
b \\
c \\
d \\
e
\end{bmatrix} \quad \text{in} \\[6pt]
A =&
\begin{bmatrix}
(x_{j_1}- x_k)^2 &  (x_{j_1} - x_k) \cdot (y_{j_1} - y_k) & (y_{j_1} - y_k)^2 & (x_{j_1} - x_k) & (y_{j_1} - y_k) \\
(x_{j_2} - x_k)^2 & (x_{j_2} - x_k) \cdot (y_{j_2} - y_k) & (y_{j_2} - y_k)^2 & (x_{j_2} - x_k) & (y_{j_2} - y_k) \\
\vdots & \vdots & \vdots & \vdots & \vdots \\
(x_{j_{n_k}} - x_k)^2 & (x_{j_{n_k}} - x_k) \cdot (y_{j_{n_k}} - y_k) & (y_{j_{n_k}} - y_k)^2 & (x_{j_{n_k}} - x_k) & (y_{j_{n_k}} - y_k) \\
\end{bmatrix}
\end{aligned}
\end{equation*} 
Seveda to rešujemo z metodo najmanjših kvadratov, kjer pa moramo predpostaviti,
da je točk znotraj radija dovolj, torej $|J_k| \geq 5$.
\end{enumerate}

\subsection{Postopek}
Sedaj lahko opišemo postopek interpolacije točk v $P$, ki poteka v treh korakih
\begin{enumerate}
\item V vsaki točki $p_k \in P$ ocenimo parcialne odvode.
\item Trianguliramo točke $(x_i,y_i)_{i=1}^n$ z neko triangulacijsko metodo.
\item Na vsakem trikotniku $T$ konstruiramo lokalno shemo z metodo Goodman-Said in shranimo matriko 
koeficientov, definiranih v prvem poglavju
\begin{equation*}
B_T = 
\begin{vmatrix}
b_{300} & b_{210} & b_{120} & b_{030} \\
b_{201} & \openbox & b_{021} & \openbox  \\
b_{102} & b_{012} & \openbox & b_{1112} \\
b_{003} & \openbox & b_{1113} & b_{1111}
\end{vmatrix}
\end{equation*}
\end{enumerate}
Seznam  matrik $B_T$ nad vsakem trikotniku triangulacije $T$, skupaj z njo
definirajo naš zlepek.

\end{document}
